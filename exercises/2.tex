\documentclass[leqno, 12pt]{article}


\usepackage[brazil]{babel} 
\usepackage[utf8]{inputenc}
\usepackage[a4paper, margin=2cm]{geometry}
\usepackage{amsfonts}
\usepackage{amsmath}
\usepackage{amssymb}
\usepackage{latexsym}
\usepackage{graphicx}
\usepackage{amsthm}
\usepackage{mathrsfs}
\usepackage{url}
\usepackage{cancel} 
\usepackage[inline, shortlabels]{enumitem}
\usepackage{xifthen} 
\usepackage{tikz}
\usepackage{listings}
\usepackage{xcolor}
\usepackage{float}
\usepackage{hyperref}
\usepackage{mathpazo}
\usepackage[style=authoryear]{biblatex}

\usetikzlibrary{automata,arrows,positioning,calc}

% \setlength{\parindent}{12 pt}

\newenvironment{sol}
{
    \vspace{4mm}
    \noindent\textbf{Resolução:}
    \strut\newline
    \smallskip
    \hspace{-3.5mm}
}
% Objetos que aparecem *após* o ambiente; 
% nestas configurações, estamos desenhando uma 
% linha horizontal. 
% (você pode, por exemplo, modificar 
% ou remover este elemento gráfico) 
{\noindent\rule{4cm}{.1mm}}

% Tamanho dos símbolos matemáticos
% \DeclareMathSizes{12}{12}{12}{9}

\addbibresource{references.bib}



\newtheorem{teo}{Teorema}[section] \newtheorem*{teo*}{Teorema}
\newtheorem{prop}[teo]{Proposição} \newtheorem*{prop*}{Proposição}
\newtheorem{lema}[teo]{Lemma} \newtheorem*{lema*}{Lema}
\newtheorem{cor}[teo]{Corolário} \newtheorem*{cor*}{Corolário}

\theoremstyle{definition}
\newtheorem{defi}[teo]{Definição} \newtheorem*{defi*}{Definição}
\newtheorem{exem}[teo]{Exemplo} \newtheorem*{exem*}{Exemplo}
\newtheorem{obs}[teo]{Observação} \newtheorem*{obs*}{Observação}
\newtheorem*{hipo}{Hipóteses}
\newtheorem*{nota}{Notação}

\newcommand{\ds}{\displaystyle} \newcommand{\nl}{\newline}
\newcommand{\eps}{\varepsilon} \newcommand{\ssty}{\scriptstyle}
\newcommand{\bE}{\mathbb{E}}
\newcommand{\cB}{\mathcal{B}}
\newcommand{\cF}{\mathcal{F}}
\newcommand{\cA}{\mathcal{A}}
\newcommand{\cM}{\mathcal{M}}
\newcommand{\cD}{\mathcal{D}}
\newcommand{\cN}{\mathcal{N}}
\newcommand{\cL}{\mathcal{L}}
\newcommand{\cLN}{\mathcal{LN}}
\newcommand{\bP}{\mathbb{P}}
\newcommand{\bQ}{\mathbb{Q}}
\newcommand{\bN}{\mathbb{N}}
\newcommand{\bR}{\mathbb{R}}
\newcommand{\bZ}{\mathbb{Z}}

\newcommand{\bfw}{\mathbf{w}}
\newcommand{\bfv}{\mathbf{v}}
\newcommand{\bfu}{\mathbf{u}}
\newcommand{\bfx}{\mathbf{x}}
\newcommand{\bfb}{\mathbf{b}}

\newcommand{\indep}{\perp \!\!\! \perp} %% indepence
\newcommand{\pr}{\operatorname{Pr}} %% probability
\newcommand{\vr}{\operatorname{Var}} %% variance
\newcommand{\rs}{X_1, X_2, \ldots, X_n} %%  random sample
\newcommand{\irs}{X_1, X_2, \ldots} %% infinite random sample
\newcommand{\rsd}{x_1, x_2, \ldots, x_n} %%  random sample, realised
\newcommand{\Sm}{\bar{X}_n} %%  sample mean, random variable
\newcommand{\sm}{\bar{x}_n} %%  sample mean, realised
\newcommand{\Sv}{\bar{S}^2_n} %%  sample variance, random variable
\newcommand{\sv}{\bar{s}^2_n} %%  sample variance, realised
\newcommand{\bX}{\boldsymbol{X}} %%  random sample, contracted form (bold)
\newcommand{\bx}{\boldsymbol{x}} %%  random sample, realised, contracted form (bold)
\newcommand{\bT}{\boldsymbol{T}} %%  Statistic, vector form (bold)
\newcommand{\bt}{\boldsymbol{t}} %%  Statistic, realised, vector form (bold)
\newcommand{\emv}{\hat{\theta}_{\text{EMV}}}

\newcommand{\bvecc}[2]{%
  \begin{bmatrix} #1 \\ #2  \end{bmatrix}
}
\newcommand{\bveccc}[3]{%
  \begin{bmatrix} #1 \\ #2 \\ #3  \end{bmatrix}
}





\title{Lista de Exercícios 2}

\author{Disciplina: Inferência Estatística \\
        Monitores: Eduardo Adame \& Ezequiel Braga}

\date{Agosto 2023}

\begin{document}

\maketitle

\section*{Cap. 7.6 (Invariância e Consistência do MLE)}

\begin{enumerate}

%%%%%%%%%%%%%%%%%%%%%%%%%%%%%%%%%%%%%%%%%%%%%%%%%%%%%%%%%
%%%%%%%%%%%%%%%%%%%%%% Exercício 1 %%%%%%%%%%%%%%%%%%%%%%
%%%%%%%%%%%%%%%%%%%%%%%%%%%%%%%%%%%%%%%%%%%%%%%%%%%%%%%%%

\item \textbf{\parencite[ex. 3, pág. 441]{DeGroot:2014}} Suponha que $X_1, \ldots, X_n$ formem uma amostra aleatória de uma distribuição exponencial para a qual o valor do parâmetro $\beta$ é desconhecido. Determine o Estimador de Máxima Verossimilhança (MLE) da mediana da distribuição.


%%%%%%%%%%%%%%%%%%%%%%%%%%%%%%%%%%%%%%%%%%%%%%%%%%%%%%%%%
%%%%%%%%%%%%%%%%%%%%%% Exercício 2 %%%%%%%%%%%%%%%%%%%%%%
%%%%%%%%%%%%%%%%%%%%%%%%%%%%%%%%%%%%%%%%%%%%%%%%%%%%%%%%%

\item \textbf{\parencite[ex. 4, pág. 441]{DeGroot:2014}} Suponha que a vida útil de um certo tipo de lâmpada siga uma distribuição exponencial para a qual o valor do parâmetro $\beta$ é desconhecido. Uma amostra aleatória de $n$ lâmpadas desse tipo é testada por um período de $T$ horas e o número $X$ de lâmpadas que falham durante esse período é observado, mas os momentos em que as falhas ocorreram não são registrados. Determine o Estimador de Máxima Verossimilhança (MLE) de $\beta$ com base no valor observado de $X$.

%%%%%%%%%%%%%%%%%%%%%%%%%%%%%%%%%%%%%%%%%%%%%%%%%%%%%%%%%
%%%%%%%%%%%%%%%%%%%%%% Exercício 3 %%%%%%%%%%%%%%%%%%%%%%
%%%%%%%%%%%%%%%%%%%%%%%%%%%%%%%%%%%%%%%%%%%%%%%%%%%%%%%%%

\item \textbf{\parencite[ex. 5, pág. 441]{DeGroot:2014}} Suponha que $X_1, \ldots, X_n$ formem uma amostra aleatória da distribuição uniforme no intervalo $[a, b]$, onde ambos os pontos finais $a$ e $b$ são desconhecidos. Encontre o Estimador de Máxima Verossimilhança (MLE) da média da distribuição.


%%%%%%%%%%%%%%%%%%%%%%%%%%%%%%%%%%%%%%%%%%%%%%%%%%%%%%%%%
%%%%%%%%%%%%%%%%%%%%%% Exercício 4 %%%%%%%%%%%%%%%%%%%%%%
%%%%%%%%%%%%%%%%%%%%%%%%%%%%%%%%%%%%%%%%%%%%%%%%%%%%%%%%%

\item \textbf{\parencite[ex. 11, pág. 441]{DeGroot:2014}} Suponha que $X_1, \ldots, X_n$ formem uma amostra aleatória de tamanho $n$ da distribuição uniforme no intervalo $[0, \theta]$, onde o valor de $\theta$ é desconhecido. Mostre que a sequência de Estimadores de Máxima Verossimilhança (MLE) de $\theta$ é uma sequência consistente.

%%%%%%%%%%%%%%%%%%%%%%%%%%%%%%%%%%%%%%%%%%%%%%%%%%%%%%%%%
%%%%%%%%%%%%%%%%%%%%%% Exercício 5 %%%%%%%%%%%%%%%%%%%%%%
%%%%%%%%%%%%%%%%%%%%%%%%%%%%%%%%%%%%%%%%%%%%%%%%%%%%%%%%%

\item \textbf{\parencite[ex. 12, pág. 441]{DeGroot:2014}} Suponha que $X_1, \ldots, X_n$ formem uma amostra aleatória de tamanho $n$ da distribuição exponencial com parâmetro desconhecido $\beta$. Mostre que a sequência de Estimadores de Máxima Verossimilhança (MLE) de $\beta$ é uma sequência consistente.

%%%%%%%%%%%%%%%%%%%%%%%%%%%%%%%%%%%%%%%%%%%%%%%%%%%%%%%%%
%%%%%%%%%%%%%%%%%%%%%% Exercício 6 %%%%%%%%%%%%%%%%%%%%%%
%%%%%%%%%%%%%%%%%%%%%%%%%%%%%%%%%%%%%%%%%%%%%%%%%%%%%%%%%

\item \textbf{\parencite[ex. 13, pág. 442]{DeGroot:2014}} Suponha que $X_1, \ldots, X_n$ formem uma amostra aleatória de uma distribuição para a qual a função de densidade de probabilidade é especificada a seguir:

\[ f(x|\theta) = \begin{cases}
\theta x^{\theta-1} & \text{para } 0 < x < 1, \\
0 & \text{caso contrário}.
\end{cases} \]

Mostre que a sequência dos Estimadores de Máxima Verossimilhança (MLE) de $\theta$ é uma sequência consistente.

\end{enumerate}

\section*{Cap. 8.8 (Informação de Fisher)}

\begin{enumerate}
\setcounter{enumi}{6}

%%%%%%%%%%%%%%%%%%%%%%%%%%%%%%%%%%%%%%%%%%%%%%%%%%%%%%%%%
%%%%%%%%%%%%%%%%%%%%%% Exercício 7 %%%%%%%%%%%%%%%%%%%%%%
%%%%%%%%%%%%%%%%%%%%%%%%%%%%%%%%%%%%%%%%%%%%%%%%%%%%%%%%%

\item \textbf{\parencite[ex. 5, pág. 527]{DeGroot:2014}} Suponha que uma variável aleatória $X$ possui a distribuição normal com média $0$ e variância desconhecida $\sigma^2 > 0$. Encontre a informação de Fisher $I(\sigma^2)$ em $X$.

%%%%%%%%%%%%%%%%%%%%%%%%%%%%%%%%%%%%%%%%%%%%%%%%%%%%%%%%%
%%%%%%%%%%%%%%%%%%%%%% Exercício 8 %%%%%%%%%%%%%%%%%%%%%%
%%%%%%%%%%%%%%%%%%%%%%%%%%%%%%%%%%%%%%%%%%%%%%%%%%%%%%%%%

\item \textbf{\parencite[ex. 4, pág 13]{Zheng:2020}} A distribuição de Rayleigh é definida como:

\[ f(x|\theta) = \begin{cases}
\frac{x}{\theta^2} \exp(\frac{x^2}{2\theta^2}) & \text{para } x \ge 0, \\
0 & \text{caso contrário}.
\end{cases} \]

Suponha que $X_1, \ldots, X_n$ formem uma amostra aleatória dessa distribuição. Encontre a variância assintótica do Estimador de Máxima Verossimlhança (MLE) de $\theta$.

%%%%%%%%%%%%%%%%%%%%%%%%%%%%%%%%%%%%%%%%%%%%%%%%%%%%%%%%%
%%%%%%%%%%%%%%%%%%%%%% Exercício 9 %%%%%%%%%%%%%%%%%%%%%%
%%%%%%%%%%%%%%%%%%%%%%%%%%%%%%%%%%%%%%%%%%%%%%%%%%%%%%%%%

\item \textbf{\parencite[ex. 6, pág 13]{Zheng:2020}} Suponha que $X_1, \ldots, X_n$ formem uma amostra aleatória da distribuição exponencial com f.d.p 

\[ f(x | \theta) = \frac{1}{\tau} \exp(\frac{-x}{\tau}), \ x \ge 0, \ \tau > 0. \]

\begin{enumerate}
    \item Encontre o MLE de $\tau$.

    \item Encontre a distribuição assintótica do MLE.
\end{enumerate}

\end{enumerate}

\section*{}

\nocite{*}
% \bibliographystyle{IEEEtran}
% \bibliography{references}
\printbibliography
\end{document}
